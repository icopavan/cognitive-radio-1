Time, space and frequency spectrum are three important shared resources in wireless communications. But these resources are limited and with the continuous development of new wireless technologies and the increasing number of coexisting wireless networks there is a need to share these resources more efficiently. 
In this paper we focus on sharing space by transmission power control (TPC). In a wireless network we can make a distinction between \textit{primary} and \textit{secondary} users. Primary users are licensed and have priority to use the spectrum. Secondary users are not licensed and are only allowed to use the spectrum where it is not needed by the primary users and the interference that they cause to the primary users has to be minimized. In TPC, the goal is to make the transmission range of a secondary transmitter as large as possible while causing as little interference as possible to any primary users.

To find spectrum holes and compute its optimal transmission range, a secondary transmitter needs knowledge of the propagation conditions and the presence of primary users. In the literature three general approaches exist to find spectrum holes. The first approach is \textit{spectrum sensing} \cite{sensing1,sensing2}. A problem here are the high sensitivity requirements to detect all activity of the primary users and to overcome the hidden terminal problem. Studies conducted by both the Federal Communications Commission (FCC) and Electronic Communications Committee (ECC) have recently revealed serious doubts about the reliability of existing spectrum sensing solutions and have subsequently deferred from making these techniques an obligatory part of secondary use regulations. However, at the same time the FCC recognized the value of sensing for, for instance, the TV White Spaces in the following statement: ``We continue to believe that spectrum sensing will continue to develop and improve. We anticipate that some form of spectrum sensing may very well be included in TV Band Devices \cite{fcc}.''  In parallel to local sensing, other approaches are however needed to achieve a more reliable detection of spectrum. 

Instead of local sensing, \textit{cooperative} sensing can be used. Here the sensing results of multiple secondary users are combined to make a decision on the presence of a primary transmitter. A second approach to detect spectrum holes is \textit{geo-location} \cite{geoLocDyspan}. In this approach a user consults a central database to determine which frequencies are available at his location. A third approach is the use of dedicated signals, \textit{beacons}, to communicate the availability of frequencies \cite{beacon}.

In the work of Pollin, Adams and Bahai \cite{sofie}, an alternative approach to detect primary users is presented. A flooding algorithm is used to communicate the presence of primary transmitters to the rest of the network. This approach has some advantages. Like spectrum sensing, the approach is distributed and relies on field measurements. Therefore it does not require any additional infrastructure. Furthermore, responses to changes in the environment are quicker. Unlike spectrum sensing, no expensive high sensitivity sensing techniques are needed. 

In this \textit{iterative power adjustment} (IPA) algorithm, the power of the secondary transmitter is iteratively adapted based on the estimated distance between the interference threshold contours of the primary transmitters and the secondary transmitter. The estimated distance between the contours is communicated with a flooding algorithm from the primary transmitters throughout the rest of the network. The assumption is that the secondary transmitter causes no interference if a minimum distance between the contours of the primary transmitters and the contour of the secondary transmitter is respected.

Our first contribution is to investigate the reliability of the IPA algorithm by looking at the \textit{location probability}. In ECC report 159 \cite{ecc}, the location probability is defined as ``The probability with which a Digital Terrestrial Television (DTT) receiver would operate correctly at a specific location; i.e. the probability with which the median wanted signal level is appropriately greater than a minimum required value''.  According to the report, the location probability is widely used in the planning of DTT networks to measure the quality of coverage of a transmitter. Because the location probability takes into account interference from secondary transmitters, we can use it to test the reliability of the computation of the optimal transmission power by the IPA algorithm. We look at a scenario were the goal is reliable coexistence of a primary and a secondary DTT transmitter and show that there is a correlation between the location probability and the propagation contour-contour distance. 

In our second contribution we adapt the flooding step of the IPA algorithm to decrease the communication overhead. Besides the reduced overhead, this also implies a reduction of interference. We compare the performance of the original and the adapted IPA algorithm in terms of the number of iteration steps and the location probability. The algorithms are simulated in a two-dimensional environment that employs a general path loss model.

The paper is organized as follows. In Sec.~\ref{sec:pre} we give some preliminaries regarding the choice of the path loss model used in the simulations and the choice to use a flooding algorithm. Sec.~\ref{sec:ipa} then explains the IPA algorithm, which consists of three components: Received power estimation, distance-to-contour flooding and path loss estimation. Here, we also discuss the path loss model used in the simulations. In Sec.~\ref{sec:contri1} we study the reliability of the IPA algorithm in terms of the location probability. We relate the location probability to the propagation contour-contour distance metric used in the algorithm. Next, we propose the alternative distance-to-contour flooding algorithm and compare it with the original flooding algorithm in Sec.~\ref{sec:contri2}. We look at several performance metrics such as the number of forwarding users, location probability and number of iterations of the IPA algorithm. Conclusions are presented in Sec.~\ref{sec:conc}. 

%